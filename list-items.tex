\documentclass{article}

\usepackage[utf8]{inputenc}
\usepackage{t1enc}
\usepackage[magyar]{babel}
\sloppy

\usepackage{amsmath}
\usepackage{amssymb}
\usepackage[only,lightning]{stmaryrd}
\usepackage{halloweenmath}
\usepackage{graphics}
\usepackage{here}
%\usepackage{MnSymbol}
%\usepackage{arev}
%\usepackage{bclogo}

\newcommand{\nothing}{\text{\raisebox{0.4em}{\rotatebox{180}{$\curvearrowleft$}}}}%{\mathbf{nothing}}
\newcommand{\just}[1]{\boxed{#1}}%{\mathbf{just}}
\DeclareMathOperator{\dom}{\mathbf{Dom}}
\newcommand{\incl}{\mathbf{incl}}
\newcommand{\excl}{\mathbf{excl}}
\newcommand{\parenthetical}[1]{\left(#1\right)}
\newcommand{\angled}[1]{\left\langle#1\right\rangle}
\newcommand{\mainfun}[3]{\mathwitch_{#1}^{#2}#3}
\newcommand{\yesmainfun}[3]{\bigpumpkin_{#1}^{#2}#3}
\newcommand{\nomainfun}[3]{\bigskull_{#1}^{#2}#3}
\newcommand{\currymainfun}[1]{\mathwitch_{#1}}
\newcommand{\storm}{\mathcloud\mspace{-42mu}\lightning\mspace{3mu}\lightning\mspace{20mu}}

\title{100. feladvány}
\author{Endrey Márk}

\begin{document}
	\maketitle

	\section{Bevezetés}

	\begin{itemize}
		\item 101: Ha 101 bárányt látsz, társad csakis 0-t láthat, és a ladbdát te biztos nem tőle, hanem a királytól kaptad. És ne is engedd át neki a labdát, hanem a királynak: azzonnal jelentsd be hogy a királylány a nyájban van.
		\item 0: Ha   0 bárányt látsz, társad 101-et vagy 100-at láthat. Ha ő volt már, akkor biztos, hogy nem 101-et, ekkor jelents be, hogy a királylány elkóborolt. Ha ő nem volt még, engedd át neki a labdát. Ha 101 báránya van, tőled függetlenül be fogja jelenteni, hogy a királynő a nyájban van, de ha 100 báránya van, akkor  épp a te átengedésed tényéből fogja megtudni, hogy te 0 bárányt látsz, ő erre alapozva be fogja jelenteni, hogy a királylány ekóborolt.
		\item 100: Ha 100 bárányt látsz, társad 0-t vagy 1-et láthat. Mindenképp engedd át számára a labdát (akár tőle, akár a királytól kaptad). Ha ő 0-t lát, be fogja jelenteni, hogy a királylány elkóborolt. Ha ő 1-et lát, vissza fogja neked dobni a labdát. Ekkor jelents be hogy a királylány a nyájban van.
		\item 1: Ha 1 bárányt lász, társad 100-at vagy 99-et láthat. Engedd áát neki a labdát. ugyanis ha ő 100-at lát, kíváncsi lesz rá, hogy te 0-t vagy 1-et látsz-e, így vissza fogja dobni neked.
	\end{itemize}

	\section{Formalizmus}

	A kommunikáció nyelvtana: a \emph{protokollfüggvény}. A $\mainfun0{-\infty}5$ függvényt úgy fogjuk felépíteni, hogy \emph{jóslóerejű} legyen: egyáltalán ($+\infty$), sőt a megadott korlátokon belül is ($2\frac12, 2, 1\frac12, 1, \frac12, 0, -\infty$). Értelmezési tartományát az $\yesmainfun0{-\infty}5$ \emph{élőségi predikátum} határolja be, ennek ellentéte, logikai negáltja a $\nomainfun0{-\infty}5$ \emph{lehetetlenségi predikátum}. Értékkészlete Maybe-algebrájú: $\nothing$ mint továbbengedés, vagy azonnali válasz $\just\incl$ és $\just\excl$ lehetőséggel. A protokollfüggvény tulajdonsága, hogy tagjaira szimmetrikus ($0, 1, 2, 3,4$) és rekurziója egyfajta komplementaritásra, illetve annak indeterminizmusára, ,,holtjátékára'' épül ($0 \mapsto 4, 5 \bullet \dots \bullet 4 \mapsto 0, 1$).

	\[\mainfun0{-\infty}5, \yesmainfun0{-\infty}5, \nomainfun0{-\infty}5\]

	\begin{align*}
		\mainfun0{-\infty}0 &= \nothing \\
		\mainfun0{-\infty}1 &= \nothing \\
		\mainfun0{-\infty}2 &= \nothing \\
		\mainfun0{-\infty}3 &= \nothing \\
		\mainfun0{-\infty}4 &= \nothing \\
		\mainfun0{-\infty}5 &= \nothing
	\end{align*}

	\begin{align*}
		\mainfun0{0}0  &= \nothing    \text{, mert }\begin{cases}\mainfun2{-\infty}4 = \nothing\\\mainfun2{-\infty}5 = \nothing&\end{cases} \\
		\mainfun0{0}1  &= \nothing    \text{, mert }\begin{cases}\mainfun2{-\infty}3 = \nothing\\\mainfun2{-\infty}4 = \nothing&\end{cases} \\
		\mainfun0{0}2  &= \nothing    \text{, mert }\begin{cases}\mainfun2{-\infty}2 = \nothing\\\mainfun2{-\infty}3 = \nothing&\end{cases} \\
		\mainfun0{0}3  &= \nothing    \text{, mert }\begin{cases}\mainfun2{-\infty}1 = \nothing\\\mainfun2{-\infty}2 = \nothing&\end{cases} \\
		\mainfun0{0}4  &= \nothing    \text{, mert }\begin{cases}\mainfun2{-\infty}0 = \nothing\\\mainfun2{-\infty}1 = \nothing&\end{cases} \\
		\mainfun0{0}5  &= \just \incl \text{, mert }\begin{cases}\mainfun2{-\infty}0 = \nothing&\text{--- tehát egyértelmű: csak egy eset van}\end{cases}
	\end{align*}

	\begin{align*}
		\mainfun0{\frac12}0  &= \just \excl \text{, mert }\begin{cases}\mainfun2{0}4 = \nothing\\\mainfun2{0}5 = \just \incl&\text{--- így nem jut el hozzá}\end{cases} \\
		\mainfun0{\frac12}1  &= \nothing    \text{, mert }\begin{cases}\mainfun2{0}3 = \nothing\\\mainfun2{0}4 = \nothing&\end{cases} \\
		\mainfun0{\frac12}2  &= \nothing    \text{, mert }\begin{cases}\mainfun2{0}2 = \nothing\\\mainfun2{0}3 = \nothing&\end{cases} \\
		\mainfun0{\frac12}3  &= \nothing    \text{, mert }\begin{cases}\mainfun2{0}1 = \nothing\\\mainfun2{0}2 = \nothing&\end{cases} \\
		\mainfun0{\frac12}4  &= \nothing    \text{, mert }\begin{cases}\mainfun2{0}0 = \nothing\\\mainfun2{0}1 = \nothing&\end{cases} \\
		\mainfun0{\frac12}5  &= \just \incl \text{, mert }\begin{cases}\mainfun2{0}0 = \nothing&\text{--- az egyetlen eset el is jut hozzá: $\angled{\frac12,5}\in\dom\currymainfun0$, azaz $\yesmainfun0{\frac12}5$}\end{cases}
	\end{align*}

	\begin{align*}
		\mainfun010  &= \just \excl \text{, mert }\begin{cases}\mainfun2{\frac12}4 = \nothing\\\mainfun2{\frac12}5 = \just \incl&\text{--- így nem jut el hozzá}\end{cases} \\
		\mainfun011  &= \nothing    \text{, mert }\begin{cases}\mainfun2{\frac12}3 = \nothing\\\mainfun2{\frac12}4 = \nothing&\end{cases} \\
		\mainfun012  &= \nothing    \text{, mert }\begin{cases}\mainfun2{\frac12}2 = \nothing\\\mainfun2{\frac12}3 = \nothing&\end{cases} \\
		\mainfun013  &= \nothing    \text{, mert }\begin{cases}\mainfun2{\frac12}1 = \nothing\\\mainfun2{\frac12}2 = \nothing&\end{cases} \\
		\mainfun014  &= \nothing    \text{, mert }\begin{cases}\mainfun2{\frac12}0 = \just\excl\\\mainfun2{\frac12}1 = \nothing&\end{cases} \\%\intertext{\[\storm\]}
		\nomainfun015 & \text{, mert }\begin{cases}\mainfun2{\frac12}0 = \just\excl&\text{--- nem jut el hozzá, így $\angled{1,5}\notin\dom \currymainfun0$}\end{cases}
		\intertext{\vspace{-1em}\[\storm\storm\storm\storm\storm\storm\]}
		%{\mathcloud\mspace{-42mu}\lightning\mspace{3mu}\lightning}\mspace{20mu}\nexists\mainfun015\bigskull\bigskull  & \text{, mert }\begin{cases}\mainfun2{\frac12}0 = \just\excl&\text{--- nem jut el hozzá, így $\angled{\frac12,5}\notin\dom P_0$}\end{cases}
	\end{align*}

	\subsection{Táblázat}

	\begin{table*}
		\caption{TODO: pont a transzponáltja kell!}
		\centering
		\begin{tabular}{c||c|c|c|c|c|c|c|}
				&	$-\infty$	&	$0$	&	$\frac12$	&	$1$	&	$1\frac12$	&	$2$	&	$2\frac12$\\\hline\hline
			$0$	&	&\nothing	&	\nothing	&	\nothing	&	\nothing	&	\nothing	&	\nothing\\\hline
			$1$	&	&\nothing	&	\nothing	&	\nothing	&	\nothing	&	\nothing	&	\nothing\\\hline
			$2$	&	&\nothing	&	\nothing	&	\nothing	&	\nothing	&	\nothing	&	\nothing\\\hline
			$3$	&	&\nothing	&	\nothing	&	\nothing	&	\nothing	&	\nothing	&	\nothing\\\hline
			$4$	&	&\nothing	&	\nothing	&	\nothing	&	\nothing	&	\nothing	&	\nothing\\\hline
			$5$	&	&\nothing	&	\nothing	&	\nothing	&	\nothing	&	\nothing	&	\nothing\\\hline
		\end{tabular}
	\end{table*}

	\subsection{Rekurziós gráf}

	\section{Tesztelő keretrendszer aktorszereposztása}
\end{document}
