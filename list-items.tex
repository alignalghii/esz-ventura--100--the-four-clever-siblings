\documentclass{article}

\usepackage[utf8]{inputenc}
\usepackage{t1enc}
\usepackage[magyar]{babel}
\sloppy

\usepackage{amsmath}
\usepackage{amssymb}
\usepackage[only,lightning]{stmaryrd}
\usepackage{halloweenmath}
\usepackage{graphics}
\usepackage{here}
\usepackage{caption}
%\usepackage{MnSymbol}
%\usepackage{arev}
%\usepackage{bclogo}
\usepackage[table]{xcolor}
\newcommand{\blk}{\cellcolor{darkgray}}
\newcommand{\gry}{\cellcolor{lightgray}}
\newcommand{\red}{\cellcolor{red!33}}
\newcommand{\grn}{\cellcolor{green!33}}

\newcommand{\nothing}{\text{\raisebox{0.4em}{\rotatebox{180}{$\curvearrowleft$}}}}%{\mathbf{nothing}}
\newcommand{\just}[1]{\boxed{#1}}%{\mathbf{just}}
\DeclareMathOperator{\dom}{\mathbf{Dom}}
\newcommand{\incl}{\mathbf{incl}}
\newcommand{\excl}{\mathbf{excl}}
\newcommand{\parenthetical}[1]{\left(#1\right)}
\newcommand{\angled}[1]{\left\langle#1\right\rangle}

\newcommand{\mainfun}[3]{\mathwitch_{#1}^{#2}#3}
\newcommand{\yesmainfun}[3]{\bigpumpkin_{#1}^{#2}#3}
\newcommand{\nomainfun}[3]{\bigskull_{#1}^{#2}#3}
\newcommand{\currymainfun}[1]{\mathwitch_{#1}}

\newcommand{\mainfunabs}[3]{\overbat{\mathwitch*_{#1}^{#2}#3}}
\newcommand{\yesmainfunabs}[3]{\overbat{\bigpumpkin_{#1}^{#2}#3}}
\newcommand{\nomainfunabs}[3]{\overbat{\bigskull_{#1}^{#2}#3}}
\newcommand{\currymainfunabs}[1]{\overbat{\mathwitch*_{#1}}}

\newcommand{\storm}{\mathcloud\mspace{-42mu}\lightning\mspace{3mu}\lightning\mspace{20mu}}

\newcommand{\progr}[1]{\left|#1\right|}

\setcounter{tocdepth}4

\usepackage{multirow}
\usepackage{array}
\setlength{\extrarowheight}{4px}

\usepackage{comment}

\title{100. feladvány}
\author{Endrey Márk}

\begin{document}
	\maketitle

	\tableofcontents

	\section{Bevezetés}

	\subsection{Tudatemlélettel rendelkező aktorok}

	\subsubsection{,,Tudom, hogy tudod, hogy tudom, hogy tudod\dots''}

	\begin{itemize}
		\item 101: Ha 101 bárányt látsz, társad csakis 0-t láthat, és a ladbdát te biztos nem tőle, hanem a királytól kaptad. És ne is engedd át neki a labdát, hanem a királynak: azzonnal jelentsd be hogy a királylány a nyájban van.
		\item 0: Ha   0 bárányt látsz, társad 101-et vagy 100-at láthat. Ha ő volt már, akkor biztos, hogy nem 101-et, ekkor jelents be, hogy a királylány elkóborolt. Ha ő nem volt még, engedd át neki a labdát. Ha 101 báránya van, tőled függetlenül be fogja jelenteni, hogy a királynő a nyájban van, de ha 100 báránya van, akkor  épp a te átengedésed tényéből fogja megtudni, hogy te 0 bárányt látsz, ő erre alapozva be fogja jelenteni, hogy a királylány ekóborolt.
		\item 100: Ha 100 bárányt látsz, társad 0-t vagy 1-et láthat. Mindenképp engedd át számára a labdát (akár tőle, akár a királytól kaptad). Ha ő 0-t lát, be fogja jelenteni, hogy a királylány elkóborolt. Ha ő 1-et lát, vissza fogja neked dobni a labdát. Ekkor jelents be hogy a királylány a nyájban van.
		\item 1: Ha 1 bárányt lász, társad 100-at vagy 99-et láthat. Engedd áát neki a labdát. ugyanis ha ő 100-at lát, kíváncsi lesz rá, hogy te 0-t vagy 1-et látsz-e, így vissza fogja dobni neked.
	\end{itemize}

	\section{Egyszerűbb feladat kitűzése tapasztalatgyűjtés céljából --- Felmerülő redukciós szempontok, ötletek}


	\begin{itemize}
		\item Csökkentjük a ,,kör'' méretét (4 helyett 3)
		\item Csökkentjük a birkák számát ($100 + 1$ helyett $4 +1$, ahol végső esetben megegyezik a kör méretével. Bár ez is nehéz, de ez már plauzibilis kihívás, jobban el tudi indulni az ember, főleg, hogy még további egérút is van: $3 +1$, $2 + 1$, $1 + 1$, és végső egyszerűsítésképp $0 + 1$, ami a lehetséges elképzelhető minimum.
	\end{itemize}

	\subsection{Körredukció}


	Meglepő módon a ,,4 helyett 3'' viszont egyelőre nem tűnik számomra tapasztatszerzsés szempontjából  előrelépésnek, redukció szempontjából meg  visszalépésnek: a 3-as eset akár még bonyolultabb is lehet következtetéek szempontjából. Egyfajta ,,4 helyett 2''-ról ugyan lehet beszélni, de az a ,,$100 + 1$ helyett'' megközelítésből adódik nagyon szerves, arra külön jellemző okokból.

	A továbbiakban nagyrészt e létszámredukció lesz érvényben az írás folyamán, és az általánosítások is e mentén történnek.
	A végén talán még visszatérünk a körredukciós, és főleg a köráltalánosítási kérdésekre is.

	\subsection{Létszámredukció}

	A $100 + 1$ helyett $4 + 1$ redukció lesz érvényes az írás nagy részében, mindenütt, ahol értelemszerűen látszik (mindenütt, ahol a 4-esek és 5-ösök szerepe kitüntetettként szerepel).

	\subsection{A redukció redukciója: elő-előtanumányok}

	\subsubsection{$4 + 1$ helyett $0 +1$}
	\subsubsection{$4 + 1$ helyett $1 +1$}
	\subsubsection{$4 + 1$ helyett $2 +1$}
	\subsubsection{$4 + 1$ helyett $3 +1$}


	\section{A párkommunikáció-elvonatkoztatás}

	\subsection{Repeater-szabályok}

	Egy későbbi formalizmust előrebocsátva, ,,repeater-szabályok'': $\mainfun1jn = \nothing$ ha $\mainfun0jn = \nothing$, és $\nomainfun1jn$ ha $\nomainfun0jn \lor \mainfun0jn = \just\incl \lor \mainfun0jn = \just\excl$.

	\section{Formalizmus}

	A kommunikáció nyelvtana: a \emph{protokollfüggvény}. A $\mainfun0{-\infty}5$ függvényt úgy fogjuk felépíteni, hogy \emph{jóslóerejű} legyen: egyáltalán ($+\infty$), sőt a megadott korlátokon belül is ($2\frac12, 2, 1\frac12, 1, \frac12, 0, -\infty$). Értelmezési tartományát az $\yesmainfun0{-\infty}5$ \emph{élőségi predikátum} határolja be, ennek ellentéte, logikai negáltja a $\nomainfun0{-\infty}5$ \emph{lehetetlenségi predikátum}. Értékkészlete Maybe-algebrájú: $\nothing$ mint továbbengedés, vagy azonnali válasz $\just\incl$ és $\just\excl$ lehetőséggel. A protokollfüggvény tulajdonsága, hogy tagjaira szimmetrikus ($0, 1, 2, 3,4$) és rekurziója egyfajta komplementaritásra, illetve annak indeterminizmusára, ,,holtjátékára'' épül ($0 \mapsto 4, 5 \bullet \dots \bullet 4 \mapsto 0, 1$).

	\[\mainfun0{-\infty}5, \yesmainfun0{-\infty}5, \nomainfun0{-\infty}5\]

	\begin{align*}
		\mainfun0{-\infty}0 &= \nothing \\
		\mainfun0{-\infty}1 &= \nothing \\
		\mainfun0{-\infty}2 &= \nothing \\
		\mainfun0{-\infty}3 &= \nothing \\
		\mainfun0{-\infty}4 &= \nothing \\
		\mainfun0{-\infty}5 &= \nothing
	\end{align*}

	\begin{align*}
		\mainfun0{0}0  &= \nothing    &&\text{, mert }\begin{cases}\mainfun2{-\infty}4 = \nothing\\\mainfun2{-\infty}5 = \nothing&\end{cases} \\
		\mainfun0{0}1  &= \nothing    &&\text{, mert }\begin{cases}\mainfun2{-\infty}3 = \nothing\\\mainfun2{-\infty}4 = \nothing&\end{cases} \\
		\mainfun0{0}2  &= \nothing    &&\text{, mert }\begin{cases}\mainfun2{-\infty}2 = \nothing\\\mainfun2{-\infty}3 = \nothing&\end{cases} \\
		\mainfun0{0}3  &= \nothing    &&\text{, mert }\begin{cases}\mainfun2{-\infty}1 = \nothing\\\mainfun2{-\infty}2 = \nothing&\end{cases} \\
		\mainfun0{0}4  &= \nothing    &&\text{, mert }\begin{cases}\mainfun2{-\infty}0 = \nothing\\\mainfun2{-\infty}1 = \nothing&\end{cases} \\
		\mainfun0{0}5  &= \just \incl &&\text{, mert }\begin{cases}\mainfun2{-\infty}0 = \nothing&\text{--- tehát egyértelmű: csak egy eset van}\end{cases}
	\end{align*}

	\begin{align*}
		\mainfun0{\frac12}0  &= \just \excl &&\text{, mert }\begin{cases}\mainfun2{0}4 = \nothing\\\mainfun2{0}5 = \just \incl&\text{--- így nem jut el hozzá}\end{cases} \\
		\mainfun0{\frac12}1  &= \nothing    &&\text{, mert }\begin{cases}\mainfun2{0}3 = \nothing\\\mainfun2{0}4 = \nothing&\end{cases} \\
		\mainfun0{\frac12}2  &= \nothing    &&\text{, mert }\begin{cases}\mainfun2{0}2 = \nothing\\\mainfun2{0}3 = \nothing&\end{cases} \\
		\mainfun0{\frac12}3  &= \nothing    &&\text{, mert }\begin{cases}\mainfun2{0}1 = \nothing\\\mainfun2{0}2 = \nothing&\end{cases} \\
		\mainfun0{\frac12}4  &= \nothing    &&\text{, mert }\begin{cases}\mainfun2{0}0 = \nothing\\\mainfun2{0}1 = \nothing&\end{cases} \\
		\mainfun0{\frac12}5  &= \just \incl &&\text{, mert }\begin{cases}\mainfun2{0}0 = \nothing&\text{--- az egyetlen eset el is jut hozzá: $\angled{\frac12,5}\in\dom\currymainfun0$, azaz $\yesmainfun0{\frac12}5$}\end{cases}
	\end{align*}

	\begin{align*}
		\mainfun010  &= \just \excl &&\text{, mert }\begin{cases}\mainfun2{\frac12}4 = \nothing\\\mainfun2{\frac12}5 = \just \incl&\text{--- így nem jut el hozzá}\end{cases} \\
		\mainfun011  &= \nothing    &&\text{, mert }\begin{cases}\mainfun2{\frac12}3 = \nothing\\\mainfun2{\frac12}4 = \nothing&\end{cases} \\
		\mainfun012  &= \nothing    &&\text{, mert }\begin{cases}\mainfun2{\frac12}2 = \nothing\\\mainfun2{\frac12}3 = \nothing&\end{cases} \\
		\mainfun013  &= \nothing    &&\text{, mert }\begin{cases}\mainfun2{\frac12}1 = \nothing\\\mainfun2{\frac12}2 = \nothing&\end{cases} \\
		\mainfun014  &= \just\incl  &&\text{, mert }\begin{cases}\mainfun2{\frac12}0 = \just\excl\\\mainfun2{\frac12}1 = \nothing&\text{--- nem jut el hozzá}\end{cases} \\%\intertext{\[\storm\]}
		&\nomainfun015              &&\text{, mert}_{\frac12}\begin{cases}\mainfun2{\frac12}0 = \just\excl&\text{--- nem jut el hozzá, így $\angled{1,5}\notin\dom \currymainfun0$}\end{cases}\\
		&                           &&\text{, mert}_1\mainfun005 = \just\incl\text{ --- alternatív bizonyítás}\\
		&                           &&\text{\hspace{7em}mostantól külön szabály}_1 \text{ az eddigi szabály}_\frac12 \text{mellé}\\
		\intertext{\vspace{-1em}\[\storm\storm\storm\storm\storm\storm\]}
		%{\mathcloud\mspace{-42mu}\lightning\mspace{3mu}\lightning}\mspace{20mu}\nexists\mainfun015\bigskull\bigskull  & \text{, mert }\begin{cases}\mainfun2{\frac12}0 = \just\excl&\text{--- nem jut el hozzá, így $\angled{\frac12,5}\notin\dom P_0$}\end{cases}
	\end{align*}

	\begin{align*}
		&\nomainfun0{1\frac12}0              &&\text{, mert}_{\frac12}\begin{cases}\mainfun214 = \just\incl&\text{--- innen már nem juthat el hozzá}\\\nomainfun215&\text{--- ez meg már nincs is a lehetőségek között}\end{cases} \\
		&                                    &&\text{, mert}_1\mainfun0{\frac12}0 = \just\excl\\
		\mainfun0{1\frac12}1  &= \just\excl  &&\text{, mert }\begin{cases}\mainfun213 = \nothing\\\mainfun214 = \just\incl&\text{--- innen nem juthat tovább}\end{cases} \\
		\mainfun0{1\frac12}2  &= \nothing    &&\text{, mert }\begin{cases}\mainfun212 = \nothing\\\mainfun213 = \nothing&\end{cases} \\
		\mainfun0{1\frac12}3  &= \nothing    &&\text{, mert }\begin{cases}\mainfun211 = \nothing\\\mainfun212 = \nothing&\end{cases} \\
		\mainfun0{1\frac12}4  &= \just\incl  &&\text{, mert }\begin{cases}\mainfun210 = \just\excl&\text{--- nem jut el hozzá}\\\mainfun211 = \nothing&\end{cases} \\%\intertext{\[\storm\]}
		&\nomainfun0{1\frac12}5              &&\text{, mert}_{\frac12}\begin{cases}\mainfun210 = \just\excl&\text{--- nem jut el hozzá}\end{cases}\\
		&                                    &&\text{, mert}_1 \mainfun0{\frac12}5 = \just\incl
	\end{align*}

	\begin{align*}
		&\nomainfun020               &&\text{, mert}_\frac12\begin{cases}\mainfun2{1\frac12}4=\just\incl\\\nomainfun2{1\frac12}5\end{cases}\\
		&                            &&\text{, mert}_1\mainfun010=\just\excl\\
		\mainfun021   &= \just\excl  &&\text{, mert }\begin{cases}\mainfun2{1\frac12}3 = \nothing\\\mainfun2{1\frac12}4 = \just\incl&\text{--- innen nem juthat tovább}\end{cases}\\
		\mainfun022   &= \nothing    &&\text{, mert }\begin{cases}\mainfun2{1\frac12}2 = \nothing\\\mainfun2{1\frac12}3 = \nothing&\end{cases}\\
		\mainfun023   &= \just\incl  &&\text{, mert }\begin{cases}\mainfun2{1\frac12}1 = \just\excl\\\mainfun2{1\frac12}2 = \nothing&\end{cases}\\
		&\nomainfun024               &&\text{, mert}_\frac12\begin{cases}\nomainfun2{1\frac12}0\\\mainfun2{1\frac12}1 = \just\excl&\text{--- nem jut el hozzá}\end{cases}\\
		&                            &&\text{, mert}_1\mainfun014=\just\incl\text{ --- nem jut el hozzá}\\
		&\nomainfun025               &&\text{, mert}_\frac12\begin{cases}\nomainfun2{1\frac12}0\end{cases}\\
		&                            &&\text{, mert}_1\nomainfun015
	\end{align*}

	\begin{align*}
		&\nomainfun0{2\frac12}0               &&\text{, mert}_\frac12\begin{cases}\nomainfun224\\\nomainfun225\end{cases}\\
		&                                     &&\text{, mert}_1\nomainfun0{1\frac12}0\\
		&\nomainfun0{2\frac12}1               &&\text{, mert}_\frac12\begin{cases}\mainfun223 = \just\incl&\text{--- innen nem juthat tovább}\\\nomainfun224\end{cases}\\
		&                                     &&\text{, mert}_1\mainfun0{1\frac12}1 = \just\excl\\
		\mainfun0{2\frac12}2   &= \just\excl  &&\text{, mert }\begin{cases}\mainfun222 = \nothing\\\mainfun223 = \just\incl&\text{--- innen nem juthat tovább}\end{cases}\\
		\mainfun0{2\frac12}3   &= \just\incl  &&\text{, mert }\begin{cases}\mainfun221 = \just\excl&\text{--- innen nem juthat tovább}\\\mainfun222 = \nothing&\end{cases}\\
		&\nomainfun0{2\frac12}4               &&\text{, mert}_\frac12\begin{cases}\nomainfun220\\\mainfun221 = \just\excl&\text{--- nem jut el hozzá}\end{cases}\\
		&                                     &&\text{, mert}_1\mainfun0{1\frac12}4=\just\incl\text{ --- nem jut el hozzá}\\
		&\nomainfun0{2\frac12}5               &&\text{, mert}_\frac12\begin{cases}\nomainfun220\end{cases}\\
		&                                     &&\text{, mert}_1\nomainfun0{1\frac12}5
	\end{align*}

	\begin{align*}
		&\nomainfun030   &&\text{, mert}_\frac12\begin{cases}\nomainfun2{2\frac12}4\\\nomainfun2{2\frac12}5\end{cases}\\
		&                &&\text{, mert}_1\nomainfun020\\
		&\nomainfun031   &&\text{, mert}_\frac12\begin{cases}\mainfun2{2\frac12}3 = \just\incl&\text{--- innen nem juthat tovább}\\\nomainfun2{2\frac12}4\end{cases}\\
		&                &&\text{, mert}_1\mainfun021 = \just\excl\\
		&\nomainfun032   &&\text{, mert }\begin{cases}\mainfun2{2\frac12}2 = \just\excl&\text{--- innen nem juthat tovább}\\\mainfun2{2\frac12}3 = \just\incl&\text{--- innen sem juthat tovább}\end{cases}\\
		&\nomainfun033   &&\text{, mert }\begin{cases}\nomainfun2{2\frac12}1\\\mainfun2{2\frac12}2 = \just\excl&\end{cases}\\
		&\nomainfun034   &&\text{, mert}_\frac12\begin{cases}\nomainfun2{2\frac12}0\\\nomainfun2{2\frac12}1\end{cases}\\
		&                &&\text{, mert}_1\nomainfun024\\
		&\nomainfun035   &&\text{, mert}_\frac12\begin{cases}\nomainfun2{2\frac12}0\end{cases}\\
		&                &&\text{, mert}_1\nomainfun025
	\end{align*}


	\section{Relatív táblák}

	Alap (,,feles'') relatív tábla, (negyedekre) kiterjesztett relatív tábla, (egészekre) leszűkített relatív tábla. Alternatív jelölésrendszerek rájuk:
	\begin{table}[H]
		\caption*{Relatív táblák változatai részletezés szerint,\\és szóbajövő jelölési alternatívák}
		\centering
		\begin{tabular}{c|ccc}
			             &  Alap                             &  Kiterjesztett                     &  Leszűkített\\\hline\hline
			Jelölés I    &  $\currymainfun{}$                &  $\uparrow\!\!\!\currymainfun{}$       &  $\downarrow\!\!\!\currymainfun{}$\\\hline
			Jelölés II   &  $\currymainfun{}$                &  ${}^{\mathghost}\!\!\!\currymainfun{}$  &  ${}_{\mathghost}\!\currymainfun{}$\\\hline
			Jelölés III  &  ${}_{\frac12}\!\!\currymainfun{}$  &  ${}_{\frac14}\!\!\currymainfun{}$   &  ${}_1\!\!\currymainfun{}$
		\end{tabular}
	\end{table}

	\subsection{Alap-reltábla: az elvonatkoztatott model fő kommunikációs felei}

	Ezt fejezi ki a leszűkített relatív tábla:

	\begin{table}[H]
		\caption*{$\currymainfun0$ és $\currymainfun2$ relatív táblázata: a 0-s és a 2-es testvér szerepe szimmetrikus a relatív reprezentációban}
		\centering
		\begin{tabular}{c||c|c|c|c|c|c|c|}
				&	$0$		&	$1$		&	$2$		&	$3$		&	$4$		&	$5$		\\\hline\hline
		$-\infty$	&	\nothing	&	\nothing	&	\nothing	&	\nothing	&	\nothing	&	\nothing	\\\hline
			$0$	&	\nothing	&	\nothing	&	\nothing	&	\nothing	&	\nothing	&	\grn\just\incl	\\\hline
		$\frac12$	&	\red\just\excl	&	\nothing	&	\nothing	&	\nothing	&	\nothing	&	\grn\just\incl	\\\hline
			$1$	&	\red\just\excl	&	\nothing	&	\nothing	&	\nothing	&	\grn\just\incl	&	\blk		\\\hline
		$1\frac12$	&	\blk		&	\red\just\excl	&	\nothing	&	\nothing	&	\grn\just\incl	&	\blk		\\\hline
			$2$	&	\blk		&	\red\just\excl	&	\nothing	&	\grn\just\incl	&	\blk		&	\blk		\\\hline
		$2\frac12$	&	\blk		&	\blk		&	\red\just\excl	&	\grn\just\incl	&	\blk		&	\blk		\\\hline
			$3$	&	\blk		&	\blk		&	\blk		&	\blk		&	\blk		&	\blk		\\\hline
		$3\frac12$	&	\blk		&	\blk		&	\blk		&	\blk		&	\blk		&	\blk		\\\hline
		\end{tabular}
	\end{table}

	\section{Kiterjesztett reltábla: a párkommunikáció-elvonatkoztatásból való visszalépés}

	A másik két testvér táblázatának kitöltésen nehezebb, hiszen a fentebbi modellek lényegében két kommunikáló félre redukálják a sémát.
	Az \#1 és \#3 indexű testvér lényegében csak afféle ,,vezeték'' ebben,hogy ez a fajta absztrakció ehetővé váljék, szerepük implicit.
	Mindenesetre ettől függetlenül nekik is konrét táblát kell adni, sőt, nemcsak az abszolút, hanem már itt a relatív tábla terén is is.

	\subsection{A repeater-szabályok} 

	A ,,repeater-szabályok'': $\mainfun1jn = \nothing$ ha $\mainfun0jn = \nothing$, és $\nomainfun1jn$ ha $\nomainfun0jn \lor \mainfun0jn = \just\incl \lor \mainfun0jn = \just\excl$.

	\begin{table}[H]
		\caption*{$\currymainfun1$ és $\currymainfun3$ relatív táblázata: a repeater szerep elfogadása esetén}
		\centering
		\begin{tabular}{c||c|c|c|c|c|c|c|}
				&	$0$		&	$1$		&	$2$		&	$3$		&	$4$		&	$5$		\\\hline\hline
		$-\infty$	&	\nothing	&	\nothing	&	\nothing	&	\nothing	&	\nothing	&	\nothing	\\\hline
		$\frac14$	&	\nothing	&	\nothing	&	\nothing	&	\nothing	&	\nothing	&	\grn\just\incl	\\\hline
		$\frac34$	&	\red\just\excl	&	\nothing	&	\nothing	&	\nothing	&	\nothing	&	\grn\just\incl	\\\hline
		$1\frac14$	&	\red\just\excl	&	\nothing	&	\nothing	&	\nothing	&	\grn\just\incl	&	\blk		\\\hline
		$1\frac34$	&	\blk		&	\red\just\excl	&	\nothing	&	\nothing	&	\grn\just\incl	&	\blk		\\\hline
		$2\frac14$	&	\blk		&	\red\just\excl	&	\nothing	&	\grn\just\incl	&	\blk		&	\blk		\\\hline
		$2\frac34$	&	\blk		&	\blk		&	\red\just\excl	&	\grn\just\incl	&	\blk		&	\blk		\\\hline
		$3\frac14$	&	\blk		&	\blk		&	\blk		&	\blk		&	\blk		&	\blk		\\\hline
		$3\frac34$	&	\blk		&	\blk		&	\blk		&	\blk		&	\blk		&	\blk		\\\hline
		\end{tabular}
	\end{table}


	Elvileg a Dom (tök és koponya)-függvényeknek is van kiterjesztett változata.

	\subsection{A repeater-szabályok levezethetősége, implicitálódása}

	Valójában persze a repeater-szabáyloknak elég a feladat szempontjából fontos esetekben megvalósulniuk. Azokon a pontokon, amelyek a rekurzív hívásokat nem rontják el érdemben, ott megengedhető a repeater-szabályoktól való eltérés.

	\begin{table}[H]
		\caption*{A $\currymainfun0$ és $\currymainfun2$ párocska relatív táblázata\\közös táblázatba vehető a $\currymainfun1$ és $\currymainfun3$ párocska relatív táblázatával\\tehát $\currymainfun{*}$ egységes relatív táblázata}
		\centering
		\begin{tabular}{c||c|c|c|c|c|c|c|}
				&	$0$		&	$1$		&	$2$		&	$3$		&	$4$		&	$5$		\\\hline\hline
		$-\infty$	&	\nothing	&	\nothing	&	\nothing	&	\nothing	&	\nothing	&	\nothing	\\\hline
			$0$	&	\nothing	&	\nothing	&	\nothing	&	\nothing	&	\nothing	&	\grn\just\incl	\\\hline
		$\frac14$	&	\nothing	&	\nothing	&	\nothing	&	\nothing	&	\nothing	&	\grn\just\incl	\\\hline
		$\frac12$	&	\red\just\excl	&	\nothing	&	\nothing	&	\nothing	&	\nothing	&	\grn\just\incl	\\\hline
		$\frac34$	&	\red\just\excl	&	\nothing	&	\nothing	&	\nothing	&	\nothing	&	\grn\just\incl	\\\hline
			$1$	&	\red\just\excl	&	\nothing	&	\nothing	&	\nothing	&	\grn\just\incl	&	\blk		\\\hline
		$1\frac14$	&	\red\just\excl	&	\nothing	&	\nothing	&	\nothing	&	\grn\just\incl	&	\blk		\\\hline
		$1\frac12$	&	\blk		&	\red\just\excl	&	\nothing	&	\nothing	&	\grn\just\incl	&	\blk		\\\hline
		$1\frac34$	&	\blk		&	\red\just\excl	&	\nothing	&	\nothing	&	\grn\just\incl	&	\blk		\\\hline
			$2$	&	\blk		&	\red\just\excl	&	\nothing	&	\grn\just\incl	&	\blk		&	\blk		\\\hline
		$2\frac14$	&	\blk		&	\red\just\excl	&	\nothing	&	\grn\just\incl	&	\blk		&	\blk		\\\hline
		$2\frac12$	&	\blk		&	\blk		&	\red\just\excl	&	\grn\just\incl	&	\blk		&	\blk		\\\hline
		$2\frac34$	&	\blk		&	\blk		&	\red\just\excl	&	\grn\just\incl	&	\blk		&	\blk		\\\hline
			$3$	&	\blk		&	\blk		&	\blk		&	\blk		&	\blk		&	\blk		\\\hline
		$3\frac14$	&	\blk		&	\blk		&	\blk		&	\blk		&	\blk		&	\blk		\\\hline
		$3\frac12$	&	\blk		&	\blk		&	\blk		&	\blk		&	\blk		&	\blk		\\\hline
		$3\frac34$	&	\blk		&	\blk		&	\blk		&	\blk		&	\blk		&	\blk		\\\hline
		\end{tabular}
	\end{table}

	\begin{table}[H]
		\caption*{Abszolút tábla}
		\centering
		\begin{tabular}{c|c|c|c|c|c|c|c|}
			\multirow{2}{*}{Menet} &  \multirow{2}{*}{Legény}  &    \multicolumn{6}{c|}{Számolható bárányok}                           \\
			                       &                           &    0                         &    1                         &    2                         &    3                         &    4                &    5      \\\hline\hline
			\multirow{4}{*}{0}     &  0                        &    $\mainfun{*}00$           &    $\mainfun{*}01$           &    $\mainfun{*}02$           &    $\mainfun{*}03$           &    $\mainfun{*}04$          &    $\mainfun{*}05$          \\\cline{2-8}
			                       &  1                        &    $\mainfun{*}{\frac14}0$   &    $\mainfun{*}{\frac14}1$   &    $\mainfun{*}{\frac14}2$   &    $\mainfun{*}{\frac14}3$   &    $\mainfun{*}{\frac14}4$  &    $\mainfun{*}{\frac14}5$  \\\cline{2-8}
			                       &  2                        &    $\mainfun{*}{\frac12}0$   &    $\mainfun{*}{\frac12}1$   &    $\mainfun{*}{\frac12}2$   &    $\mainfun{*}{\frac12}3$   &    $\mainfun{*}{\frac12}4$  &    $\mainfun{*}{\frac12}5$  \\\cline{2-8}
			                       &  3                        &    $\mainfun{*}{\frac34}0$   &    $\mainfun{*}{\frac34}1$   &    $\mainfun{*}{\frac34}2$   &    $\mainfun{*}{\frac34}3$   &    $\mainfun{*}{\frac34}4$  &    $\mainfun{*}{\frac34}5$  \\\hline\hline
			\multirow{4}{*}{1}     &  0                        &    $\mainfun{*}10$           &    $\mainfun{*}11$           &    $\mainfun{*}12$           &    $\mainfun{*}13$           &    $\mainfun{*}04$          &    $\mainfun{*}05$          \\\cline{2-8}
			                       &  1                        &    $\mainfun{*}{1\frac14}0$  &    $\mainfun{*}{1\frac14}1$  &    $\mainfun{*}{1\frac14}2$  &    $\mainfun{*}{1\frac14}3$  &    $\mainfun{*}{\frac14}4$  &    $\mainfun{*}{\frac14}5$  \\\cline{2-8}
			                       &  2                        &    $\mainfun{*}{1\frac12}0$  &    $\mainfun{*}{1\frac12}1$  &    $\mainfun{*}{1\frac12}2$  &    $\mainfun{*}{1\frac12}3$  &    $\mainfun{*}{\frac12}4$  &    $\mainfun{*}{\frac12}5$  \\\cline{2-8}
			                       &  3                        &    $\mainfun{*}{1\frac34}0$  &    $\mainfun{*}{1\frac34}1$  &    $\mainfun{*}{1\frac34}2$  &    $\mainfun{*}{1\frac34}3$  &    $\mainfun{*}{\frac34}4$  &    $\mainfun{*}{\frac34}5$  \\\hline\hline
			\multirow{4}{*}{2}     &  0                        &    $\mainfun{*}20$           &    $\mainfun{*}21$           &    $\mainfun{*}22$           &    $\mainfun{*}23$           &    $\mainfun{*}04$          &    $\mainfun{*}05$          \\\cline{2-8}
			                       &  1                        &    $\mainfun{*}{2\frac14}0$  &    $\mainfun{*}{2\frac14}1$  &    $\mainfun{*}{2\frac14}2$  &    $\mainfun{*}{2\frac14}3$  &    $\mainfun{*}{\frac14}4$  &    $\mainfun{*}{\frac14}5$  \\\cline{2-8}
			                       &  2                        &    $\mainfun{*}{2\frac12}0$  &    $\mainfun{*}{2\frac12}1$  &    $\mainfun{*}{2\frac12}2$  &    $\mainfun{*}{2\frac12}3$  &    $\mainfun{*}{\frac12}4$  &    $\mainfun{*}{\frac12}5$  \\\cline{2-8}
			                       &  3                        &    $\mainfun{*}{2\frac34}0$  &    $\mainfun{*}{2\frac34}1$  &    $\mainfun{*}{2\frac34}2$  &    $\mainfun{*}{2\frac34}3$  &    $\mainfun{*}{\frac34}4$  &    $\mainfun{*}{\frac34}5$  \\\hline\hline
			\multirow{4}{*}{3}     &  0                        &    $\mainfun{*}30$           &    $\mainfun{*}31$           &    $\mainfun{*}32$           &    $\mainfun{*}33$           &    $\mainfun{*}04$          &    $\mainfun{*}05$          \\\cline{2-8}
			                       &  1                        &    $\mainfun{*}{3\frac14}0$  &    $\mainfun{*}{3\frac14}1$  &    $\mainfun{*}{3\frac14}2$  &    $\mainfun{*}{3\frac14}3$  &    $\mainfun{*}{\frac14}4$  &    $\mainfun{*}{\frac14}5$  \\\cline{2-8}
			                       &  2                        &    $\mainfun{*}{3\frac12}0$  &    $\mainfun{*}{3\frac12}1$  &    $\mainfun{*}{3\frac12}2$  &    $\mainfun{*}{3\frac12}3$  &    $\mainfun{*}{\frac12}4$  &    $\mainfun{*}{\frac12}5$  \\\cline{2-8}
			                       &  3                        &    $\mainfun{*}{3\frac34}0$  &    $\mainfun{*}{3\frac34}1$  &    $\mainfun{*}{3\frac34}2$  &    $\mainfun{*}{3\frac34}3$  &    $\mainfun{*}{\frac34}4$  &    $\mainfun{*}{\frac34}5$  \\\hline
		\end{tabular}
	\end{table}

	\begin{table}[H]
		\caption*{Abszolút tábla}
		\centering
		\begin{tabular}{c|c|c|c|c|c|c|c|}
			\multirow{2}{*}{Menet} &  \multirow{2}{*}{Legény}  &    \multicolumn{6}{c|}{Számolható bárányok}                                                          \\
				               &                           &    0              & 1              & 2              & 3              & 4              & 5              \\\hline\hline
			\multirow{4}{*}{0}     &  0                        &    \nothing       & \nothing       & \nothing       & \nothing       & \nothing       & \grn\just\incl \\\cline{2-8}
				               &  1                        &    \nothing       & \nothing       & \nothing       & \nothing       & \nothing       & \grn\just\incl \\\cline{2-8}
				               &  2                        &    \red\just\excl & \nothing       & \nothing       & \nothing       & \nothing       & \grn\just\incl \\\cline{2-8}
				               &  3                        &    \red\just\excl & \nothing       & \nothing       & \nothing       & \nothing       & \grn\just\incl \\\hline\hline
			\multirow{4}{*}{1}     &  0                        &    \red\just\excl & \nothing       & \nothing       & \nothing       & \grn\just\incl & \blk           \\\cline{2-8}
				               &  1                        &    \red\just\excl & \nothing       & \nothing       & \nothing       & \grn\just\incl & \blk           \\\cline{2-8}
				               &  2                        &    \blk           & \red\just\excl & \nothing       & \nothing       & \grn\just\incl & \blk           \\\cline{2-8}
				               &  3                        &    \blk           & \red\just\excl & \nothing       & \nothing       & \grn\just\incl & \blk           \\\hline\hline
			\multirow{4}{*}{2}     &  0                        &    \blk           & \red\just\excl & \nothing       & \grn\just\incl & \blk           & \blk           \\\cline{2-8}
				               &  1                        &    \blk           & \red\just\excl & \nothing       & \grn\just\incl & \blk           & \blk           \\\cline{2-8}
				               &  2                        &    \blk           & \blk           & \red\just\excl & \grn\just\incl & \blk           & \blk           \\\cline{2-8}
				               &  3                        &    \blk           & \blk           & \red\just\excl & \grn\just\incl & \blk           & \blk           \\\hline\hline
			\multirow{4}{*}{3}     &  0                        &    \blk           & \blk           & \blk           & \blk           & \blk           & \blk           \\\cline{2-8}
				               &  1                        &    \blk           & \blk           & \blk           & \blk           & \blk           & \blk           \\\cline{2-8}
				               &  2                        &    \blk           & \blk           & \blk           & \blk           & \blk           & \blk           \\\cline{2-8}
				               &  3                        &    \blk           & \blk           & \blk           & \blk           & \blk           & \blk           \\\hline
		\end{tabular}
	\end{table}

	A ,,negyedes'' legények (1, 3) kitöltéskekor érzékeljük, hogy információveszteség történik: nem figyljük, egy-egy legény mennyi bárányt számol az egyik irányban külön és a másik irányban külön.
	Ez a fajta információvesztés --- ,,a karok vesztesége'' --- a később említendő ,,oprtimalitás tétele'' szempontjából lesz érdekes.

	\section{Tételek és kérdések}

	\subsection{Az autonómia tétele és az optimalitás tétele}

	Az autónómia téetle aztmondja ki, hog az 1-es és 3-as legénynek nem kell kólon \texttt{const Nothing} jellegű szabályokat előírnunk.
	Dolgozhatnak ugyanazzal a semeatikával, mint a 0-s és 2-es legény. Persze a konrét forgatókönyvekben vanóba afféle \texttt{const Nothing} repeater jellegű szerep hárul rájuk,
	de ez automatikusan áll elŐ: egyszerűen abból fakad. hogy az ő információjuk előrehaladottsgi foka kisebb, mint az éppen érdeklődés alaptt álló legényé, így a rekurziók során ténylegesen előhívásra kerülő helyzetekben ők tényleg Nothingot nyújtnak a számára.

	Az optimalitás tétele hasonló jellegű. Attól még hogy bebizonosodik, hogy az itt tárgyalt sematika tud jó válaszokat asni a kiránynak a keretek közt, attól még elvileg lehetnek nehézségek a feladat kérdéseinek megadásában:

	\begin{quotation}
		Meg tudjátok mondani, hogy melyik legény felelt meg, hányadik kérdésre, és mit felelt?
	\end{quotation}

	Ehhez nem elég tudnunk azt, hogy mi egy jó jóslófüggvény, ehhez azt is tudnunk kell, hogy a jóslófüggvény létezése bizonyos értleemben megköti a lehetőségeket, és nem lehetsége,s hogy eg másik jóslófüggvényt használva a fiók képesek lennének megfelelni a királynak, de a feladat kérdésére más válasz adódnék.

	Előfordulhat az, hogy az optimaliás tétele nem igaz, és a feladatnak egy akár életszerűtlen de létző alternatív megoldás is van?

	Az optimaliás tételét jelenleg nem tudom bizonyítani, de a feladat impliciten kihasználja, előfeletételezi fennállását. Így a feladat szószerinti kérdésére tett választ ennek figyelembevételével közlöm.

	\subsection{MI-autonómia, emergencia?}

	A harmadik kérdés még e területen inkább kihívás, mint tétel, az autonómia még tovább vitele. Ez a négyfitút mint önálló autonóm aktort, genst még tovább viszi. Lehetséges-e olyan MI-programot írni, ahol a jóslófüggvényre a program magától jön rá, persze a királlyal való előzetes ,,dry run'' próbajászmák során? Felmutatható-e egy ilyen webalkalmazás? Ez esetben a fiúk egyöntetűen sak egy ilyen tanuló algoritmussal érkeznek. Minden máásra már dry run próbaajátékok során kell rájönni tudniuk.

	Mindenesetre programozóknak e kérdéskor jó ,,ürügyet'' ad elosztott algoritmusok, felhőwebalkalmazások gyakorlására.

	Ami viszont innen már a fizika kérdésköreihez is vezethet: emergencia, emergens rendszerek, egyéb önszabályozási kérdések, flock jelenségek. Biológiai és társadalmi példák: tömeg visekledése (diszkótűz, szurkolótömegben hullámok), biológiában a termeszek bárépítő emergenciája jóval egyszerűbb egyéni alkoritmusok alapján, amelyek az egyedekben egységes(ebb)en vannak benne.

	\section{Abszolút táblák}

	A fenti kérdések megközelítéséhez tovább kell lépnünk a modellezésben: élően meg kell valósítanunk a feladatot.
	A korét leprogramozásnál érdemes meghatároznunk a már végső kódot közvetlenül irányító abszolút táblákat.
	Nem feletétlen kell a porgramnak ezeket használnia: az autonómia tétele szerint elegendő a relatívá táblákat használnunk.
	A relatív táblákat eekkor emergens viselekedés eredményeként kapjuk, deklarálásuk tehát implicit, illetve esetleg a tesztek tartalmazzák.
	De épp ezért, akár csak tesztcélokból is, de érdemes önállóan is meghatárznunk az abszolút táblákat.

	Jelölésben a macskás boszorkány.

	\subsubsection{Az elvonatkoztatott model fő kommunikációs felei}

	\begin{table}[H]
		\caption*{$\currymainfunabs0$ abszolút táblázata}
		\centering
		\begin{tabular}{c||c|c|c|c|c|c|c|}
				&	$0$		&	$1$		&	$2$		&	$3$		&	$4$		&	$5$		\\\hline\hline
			$0$	&	\nothing	&	\nothing	&	\nothing	&	\nothing	&	\nothing	&	\grn\just\incl	\\\hline
			$1$	&	\red\just\excl	&	\nothing	&	\nothing	&	\nothing	&	\grn\just\incl	&	\blk		\\\hline
			$2$	&	\blk		&	\red\just\excl	&	\nothing	&	\grn\just\incl	&	\blk		&	\blk		\\\hline
			$3$	&	\blk		&	\blk		&	\blk		&	\blk		&	\blk		&	\blk		\\\hline
		\end{tabular}
	\end{table}

	\begin{table}[H]
		\caption*{$\currymainfunabs2$ abszolút táblázata}
		\centering
		\begin{tabular}{c||c|c|c|c|c|c|c|}
				&	$0$		&	$1$		&	$2$		&	$3$		&	$4$		&	$5$		\\\hline\hline
		$0$	&	\red\just\excl	&	\nothing	&	\nothing	&	\nothing	&	\nothing	&	\grn\just\incl	\\\hline
		$1$	&	\blk		&	\red\just\excl	&	\nothing	&	\nothing	&	\grn\just\incl	&	\blk		\\\hline
		$2$	&	\blk		&	\blk		&	\red\just\excl	&	\grn\just\incl	&	\blk		&	\blk		\\\hline
		$3$	&	\blk		&	\blk		&	\blk		&	\blk		&	\blk		&	\blk		\\\hline
		\end{tabular}
	\end{table}

	\paragraph{Egy teljes alternatív kiépítése az egésznek}
	A tábla szimmetriája miatt az egész feladat tárgyalását fel lehetne építeni matematikailag is és didaktikailag is úgy, hogy eleve  ezt a táblázatot ismerjük fel (létezik hétköznapi foglamakkal megadható leírás és rávezetés).

	\subsection{A párkommunikáció-elvonatkoztatásból való visszalépés}

	\subsubsection{A kimaradt abszolút és relatív táblák}

	\begin{table}[H]
		\caption*{$\currymainfun1$ relatív táblázata}
		\centering
	\end{table}

	\begin{table}[H]
		\caption*{$\currymainfun3$ relatív táblázata}
		\centering
	\end{table}

	\begin{table}[H]
		\caption*{$\currymainfunabs1$ abszolút táblázata}
		\centering
	\end{table}

	\begin{table}[H]
		\caption*{$\currymainfunabs3$ abszolút táblázata}
		\centering
	\end{table}

	\section{Egyéb kísérleti fogalmak bölcseje}

	\subsection{Progressziós/előrehaladottsági érték}
	\begin{displaymath}
		\progr{\text{fehér}} < \progr{\text{színes}} < \progr{\text{fekete}}
	\end{displaymath}
	vagyis
	\begin{align*}
		\progr{\mathwitch=\nothing} < &\progr{\mathwitch=\just\incl} < \progr{\bigskull}\\
		                            < &\progr{\mathwitch=\just\excl} <
	\end{align*}
	\subsection{Szabályok}

	\subsection{Rekurziós gráf}

	Szintezettség, hasonló a táblázathoz, de az indeterminisztikus komplementaritás miatt az élek kuszák a szintek között.

	\section{Tesztelő keretrendszer aktorszereposztása}

	Van a négy azonos típusú és viselkedésű aktor, akik közös protokollfügyvény szerint viselkednek, és van a király mint egy magasabb szinten vett másik fél.

	Érdekesség még, hogy a négy testvéraktor esetében is valójában a szembelevők közt zajlik az érdemi kommunikáció. Lehet, hogy létezik optimálisabb protokoll is (amely nem egyszerűűsíti le a négy testvérktort kettőre, és ezt kihasználva hatékonyabb), de a protokoll optimalitása nem feladat, csak a korlátokon belül maradása.

	\section{Hálózati alkalmazások, játékok}

	Az 1-játékosú játék, még szoliternek sem nevezhető, hiszen itt az egyetlen játékos szerepét a gép játssza, külső játékos (felhasználó) nincs is. Webalkalmazásképp nem is interaktivitási okokból érdemes megcsinálni, legfeljebb adott paraméterezésű (pl.~100-birkás, 50-50-50-50-partíciós) játék stratégiáinak teszteléseként.

	A 2-játékosú játék már érdekes, itt van egy külső játékos: vagy a király, vagy a testvérnégyes. Itt azonban didaktikai és játékfejlesztési jelentősége egy szereposztás-megválasztható webalkalmazásnak van. A felhasználó először a testvérnégyes szerepét játszva mérközik a királlyal (a géppel). Ebből tanulva, gyakorolva felfedezve láthat neki a fordított szereposztásnak.

	A 3, 4, 5 játékosú játékok szzimetrikusan: a testvrnégyest bontjuk fel. A 6-játékosú játék teljsen fölvan bontva. Mikroszervizek szempontjából lehetnek érdekesek.

	Érdekesek lehetnek olyan játékok, ahol ugyan több gépi vagy több emberi játékos is van, de ezek mikroszervizesen, hálózati úton tartják a kapcsolatot. Midenesetre programozás gyakorlása szempontjából érdekes, játék szempontjából még nem látom hasznukat, bár lehet hogy van ilyen. Didakitukailag érdekes lehet, hogy több gyerek játszhat egyszerre, pl. a néhy testvér szerepében rendre, esetleg közülük kettőt vagy hármat vagy mind a négyet ,,tömbösítve''.

	Olyan játék, ahol a birkák összlétszámát, vagy a kör méretét a játékos nem is tudja, és neki kell kitalálnia (esetleg több élet engedélyezésével?) Van-e ilyen?

	A továbbiakban a kétjátékosú játék, és a szereposztásmegválasztható webalkalmazás fejlesztéséről lesz szó.

	\section{Általánosítások}

	Köráltalánosítás? (A körredukció mintáájára). Létszámáltalánosítás, illetveennek részeként partícióáltalánosítás? A király egyéb feltételei (tudjuk a látszámot, tudjuk a kör méretét, a király szigorúsága, a meggengedett körök száma).
\end{document}
